
% A crisis is unpredictable by nature. Despite this, it presents patterns that can help communicators to anticipate problems to give faster and better responses to emergency situations. The current solutions for emergency public communication focus on the dissemination of a single message through different communication channels, for all audiences. This strategy is against the good communication studies because different audiences will receive information that is not of interest to them or that will not be presented in the best way for their understanding (e.g, technical and statistical information for the lay public). An inappropriate public communication message can amplify the perception of risk from at the public,and thus, be adding to the insecurity and lead to the creation of noises in communication that are difficult to fix in the progress of the crisis. Several studies have been conducted to propose good communication principles, among which we can highlight: be first (for the public the first source is the most reliable); be right (accuracy establishes credibility); be credible (honesty and truthfulness are essential during a crisis); communicate repeatedly (whenever possible, keep the public informed); and communicate by different tools, media and communication channels (never rely on a single method of communication). Our research, considers these principles to propose a computational model for public communication of emergencies. We mapped the variability in the process of communication of emergencies in order to ensure flexibility of configuration in different scenarios and types of emergencies. The main contribution of our research is a variability based approach to support customised communication of the emergencies and its consequences targeted to proper audiences. In this approach, we defined four variability behaviours in the content of the sentences in order to support the adaption according to the emergency state information. This research is being developed inside of a larger research project named RESCUER\footnote{Rescuer Project - http://www.rescuer-project.org} -- a joint Brazil-Europe initiative, involving nine research and industry organisations in four countries (Brazil, Austria, Germany and Spain). RESCUER aims at developing an interoperable solution to support command centres in quickly managing emergencies, based on reliable and intelligent analysis of crowdsourcing information mashed up with open data.
%The term “crisis communication” is usually used in two ways: (1) to refer to communication between organisations involved in managing the crisis and (2) to refer to communication from the emergency management to inform and alert the public about the emergency. The latter definition fits best the scope of our work, as it focuses on minimising the challenges faced by communicators while informing and alerting the public about an emergency. 


A crisis is unpredictable by nature, but it presents patterns that can help communicators to anticipate problems and give better responses to emergency situations. Current solutions for emergency public communication focus on the dissemination of a single message through different communication channels. This strategy goes against good communication practices because different audiences have different information needs. Inappropriate public communication messages create noise and can amplify the perception of risk and insecurity.  Our research proposes a computational model for public communication in emergencies. It maps and models variability in emergency communication processes to ensure the flexibility and adaptability of the communication configuration for different emergency scenarios. The model aims to facilitate the rapid construction of customised and consistent emergency communications, over different channels, for different audiences. The model was developed as part of a larger research project named RESCUER, which uses crowdsourcing information to monitor and manage emergency situations. To evaluate the communication model, a proof of concept, named RESCUER News, was built and evaluated in simulations involving real public and operational forces, over three distinct emergency scenarios. 




 %As proof-of-concept we developed a solution for public communication called “RESCUER News”. The solution was then used by experts on emergency management of two countries (Brazil and Austria), tested in simulations of emergencies in two different scenarios (Industrial Parks and Large Scale Events).
\begin{keywords}
Emergency Public Communication; Crisis Communication; Document Variability; 
\end{keywords}

