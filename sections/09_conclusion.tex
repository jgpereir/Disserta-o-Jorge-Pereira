\xchapter{Conclusion}{} \label{chapter:conclusion}

Public communication is a key activity in emergency and crisis management. The challenge of this activity is to establish a good strategy for communication with the partners that should be informed. A crucial aspect is to ensure that the right people will receive the information they need, when they need it.

We specified in detail the variabilities that support the configuration to different scenarios and organisations, as well as the variabilities that allow the sending of customised public communications in a semi-automatic way, taking into account emergency state, target audience and communication channel.

Our proposed brings the benefit of using dynamics templates to allow the semi-automatic and flexible creation of public communications taking into consideration the target audience, incident type, emergency phase and adapt the content of the message according to emergency state information. For this reason, we can have been able to adapt our prof of concept solution to work in a addiction scenario, like the simulation in Lambach Tunnel, without necessity of specific customisation our code implementation.  